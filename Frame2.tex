% ----------------------------------------
% Filename : Frame2.tex
% ----------------------------------------

%
% Frame 2a)
%

\begin{frame}
\frametitle{An intriguing number : \(-1\)}

The number \(-1\) on its own might not seem all that interesting. But consider what happens if we repeatedly multiply this number by itself.

\begin{align*}
& -1 \times -1 = i^{2} = 1  \\
& -1 \times -1 \times -1 = i^{3} = -1  \\
& -1 \times -1 \times -1 \times -1 = i^{4} = 1  \\
& -1 \times -1 \times -1 \times -1 \times -1 = i^{5} = -1  \\
& -1 \times -1 \times -1 \times -1 \times -1 \times -1 = i^{6} = 1
\end{align*}

Notice how the results flip back and forth between the values of \(1\) and \(-1\). To put this another way, the results alternate between the two halves of the
real number line. This is rather intriguing! \\~\

\end{frame}


%
% Frame 2a)
%

\begin{frame}
\frametitle{A vexing number : \(\sqrt{-1}\)}

If like the author, you do indeed think that the number \(-1\) is intriguing, then wait till you have a look at \(\sqrt{-1}\).\\~\

If you're thinking to yourself -- wait a minute, you can't take the square root of a negative number, then hold those thoughts for a moment.\\~\

While you're holding those thoughts, consider the following equation and then try to solve it for \(x\).

\begin{equation}
  x^{2} + 5 = 4
\end{equation}

Re-arranging gives;

\begin{equation}
  x^{2} = 4 - 5
\end{equation}

\end{frame}


\begin{frame}
\frametitle{You can't take the square root of a negative number -- or can you?}

Thus;

\begin{equation}
  x = \pm\sqrt{-1}
\end{equation}

Hang on a moment! What's going on here? It seems like we've got ourselves into a bit of a dilemma. Most people tend to be of the belief that you can't
take the square root of a negative number. But if you can't, then how would one go about solving the equation from the previous slide?\\~\

This dilemma raises two serious questions.\\~\

\begin{enumerate}
  \item Can you even take the square root of a negative number?
  \item If you can indeed take the square root of a negative number, is the result even a real number?
\end{enumerate}

\end{frame}


%
% Frame 2b)
%

\begin{frame}
\frametitle{The problem with the real numbers.}

Before we proceed any further, let us pause for a moment and consider the real numbers.\\~\

All of the real numbers reside in their own set. You can think of this set as ``the set of real numbers''. It probably contains every unitarian -- i.e. 
non multi-part number that you can possibly think of. For example, \(0\), \(1\), \(0.07836\), \(1,000,000\), \(21/34\),
\(\pi\), \(e\) and so on. These values all reside in the set of real numbers.\\~\

But what about the number \(\sqrt{-1}\) from the previous slide? Whereabouts does it reside?\\~\

The answer is that number does not reside within the set of real numbers at all! So if it does not reside within the set of real numbers, where exactly
does it reside then?\\~\

\end{frame}


%
% Frame 2b)
%

\begin{frame}
\frametitle{Introducing the imaginary numbers.}

The answer is that it resides within a \textit{completely} different set of numbers from the one which is used to hold all of the real numbers. The values in this
completely different set of numbers are referred to as the ``imaginary numbers'', while the set itself is referred to as the ``set of imaginary numbers''.\\~\ 

THIS IS AN EXTEREMELY IMPORTANT POINT WHICH WAS JUST MADE!\\~\

Please make sure that you fully understand what it is saying.\\~\

The unit value -- or the base unit if you like, of the imaginary numbers is defined as follows;

\begin{equation}
  i = \sqrt{-1}
\end{equation}

\end{frame}


\begin{frame}
\frametitle{Some history around imaginary numbers.}

The idea of representing these numbers with the letter \(i\), was introduced by Leonhard Euler -- one of the greatest mathematicians of all time. It is believed
that he chose the letter \(i\) for the Italian word for imaginary -- that is \textit{``immaginaria''} or \textit{``immaginario''}.\\~\

Although Euler introduced the idea of using the letter \(i\), it was the mathematician and philosopher Ren\'e Descartes who is widely credited with popularizing
the term ``imaginary numbers''.\\~\

\end{frame}


%
% Frame 2c)
%

\begin{frame}
\frametitle{Reluctance toward the use of imaginary numbers.}

It is worth mentioning that many people are confused by, put off by, or simply don't like the name ``imaginary numbers''. They feel that because these numbers
are referred to as ``imaginary'', somehow they don't really exist, or are of lesser significance than real numbers.\\~\

It is not just everyday people or newcomers to the field of mathematics who feel this way about imaginary numbers either. Arguably the greatest mathematician
of all time -- that is Carl Friedrich Gauss, did not like the name ``imaginary numbers''. Instead, he suggested that they should be called ``lateral numbers'' or
``lateral units''.\\~\

\end{frame}


%
% Frame 2d)
%

\begin{frame}
\frametitle{Imaginary numbers in the real world.}

When the Nobel prize winning physicist Erwin Schr\"odinger introduced his famous wave equation in 1926 -- to explain mathematically how electrons
could behave as waves, many physicists were uncomfortable with the appearance of the \(i\) within the equation.

\begin{equation}
  i\hbar\dfrac{\partial}{\partial t}\ket{\Psi} = \hat{H}\ket{\Psi}
\end{equation}

 Why was it there, and what could its presence possibly mean in the real world?\\~\

\end{frame}


%
% Frame 2d)
%

\begin{frame}
\frametitle{Imaginary numbers and the prediction of anti-matter.}

Fortunately, another Nobel prize winning physicist by the name of Paul Dirac, soon built upon Schr\"odinger's work. Dirac offered the explanation that maybe the
\(i\) could represent a solution for some alternate form of matter. Maybe matter from another dimension, imaginary matter, or maybe 
lateral matter? Dirac even went so far as to suggest that electrons with positive charge might even exist!\\~\

Some physicists scoffed at this suggestion, but it didn't take long before another Nobel prize winning physicist -- this time Carl Anderson in 1932, confirmed
by experimentation that these positively charged electrons do indeed exist. Nowadays, we generally don't refer to such particles as positively charged electrons.
Instead, we call them ``positrons'', and they belong to a family of particles which we call ``anti-matter''.

\end{frame}


%
% Frame 2e)
%

\begin{frame}
\frametitle{Don't be afraid of imaginary numbers.}

So there you go! Don't be afraid of imaginary numbers -- or as Gauss would rather you call them, lateral numbers.\\~\

They really do exist -- and as we have just seen, they were even used to mathematically predict the existence of anti-matter particles.\\~\

\end{frame}


\begin{frame}
\frametitle{The imaginary number line.}

Just as all the values in the set of real numbers can be mapped or laid out onto the real number line, so too can all the values in the set of imaginary numbers
be mapped or laid out onto the imaginary number line.\\~\

\(\bullet\) Mapping onto the imaginary number line.\\~

But how exactly do we get into or map onto this imaginary number line from the world of real numbers? As we saw earlier, one way is to take a real number from
the real number line and multiply it by \(i\). Doing this will map us from the chosen real number to its corresponding imaginary number on the
imaginary number line.

\begin{align*}
1 \times i     &\mapsto i \\
-2 \times i    &\mapsto -i2 \\
0.073 \times i &\mapsto i0.073
\end{align*}

\end{frame}


\begin{frame}
\frametitle{The imaginary number line ... continued}

\(\bullet\) Mapping off of the imaginary number line.\\~

Now that we've seen how to map a real number onto the imaginary number line, how do we do the opposite. That is, how do we map an imaginary number onto the
real number line?\\~\

Recall that;

\begin{equation}
i = \sqrt{-1}
\end{equation}

Therefore, \(i^{2} = -1\).\\~\

So if we multiply an imaginary number by \(i\) then we will map from that imaginary number onto the negative version of the real number which corresponds to
that same imaginary number.\\~\

Confusing? Let's see some more examples on the next slide.

\end{frame}


\begin{frame}
\frametitle{The imaginary number line ... continued some more}

Let's take the results of the examples from earlier, where we mapped real numbers onto the imaginary number line. Remember that all of these results were
imaginary numbers. Just like the previous examples, we are going to multiply these values by \(i\) as well. Let's see what happens.

\begin{align*}
i \times i      &\mapsto -1 \\
-i2 \times i    &\mapsto 2 \\
i0.073 \times i &\mapsto -0.073
\end{align*}

We can see that all of the imaginary number results from earlier have been mapped onto the real number line.\\~\

So we can see that multiplying a real number by \(i\) twice in succession, is equivalent to multiplying that same real number by \(-1\).

\end{frame}


%
% Frame 2f)
%

\begin{frame}
\frametitle{Plotting the function \(f(x) = \pm\sqrt{x}\)}

Let us digress briefly before we return back to the real and imaginary number lines.\\~\

Consider the 2-dimensioanl Cartesian co-ordinate system which would be required to handle a plot of the following function;

\begin{equation}
  f(x) = \pm\sqrt{x}
\end{equation}

What would happen to the resulting plot of this function if the values for \(x\) were to go negative? Would a plot of this function even exist in this case?\\~\

The answer is both yes and no. Yes a plot would exist for this function in this scenario, but no -- it won't reside within the same 2-dimensional Cartesian
co-ordinate system within which the plot exists when \(x\) is positive.\\~\

\end{frame}


%
% Frame 2g)
%

\begin{frame}
\frametitle{Solutions to these problems.}

So where would this plot exist then?\\~\

The answer is that it exists in an entirely different 2-dimensional co-ordinate system which is referred to as a complex, Argand, or Gauss plane.

\includegraphics[scale=0.25]{"Plot of square root of x.png"}

\end{frame}


%
% Frame 2h)
%

\begin{frame}
\frametitle{1-d oscillations on the real number line.}

We have seen from the previous slides, that it is a real number \(-1\), which helps to form the basis for the imaginary number \(i\).\\~\

Let us now pause for a moment and give some further consideration to the real number \(-1\).\\~\

As we saw earlier, when this number is repeatedly multiplied by itself, the resulting values simply alternate between the two values of \(1\) and \(-1\).
If n is odd, then the result will be negative; whereas if n is even, then the result will be positive.\\~\

This exhibits movement in a 1-d line.

\end{frame}


%
% Frame 2i)
%

\begin{frame}
\frametitle{2-d rotations in the complex plane.}

Let us try and build on this alternating behaviour from the previous slide and see if we can graduate from 1-d alternating behaviour on a line, to 2-d
rotational behaviour in a plane.\\~\

Consider the following function.

\begin{equation}
  f(x) = \pm\sqrt{x}
\end{equation}

What will happen if we try and plot the function for negative values of x. The Cartesian co-ordinate system won't suffice, as it doesn't incorporate
imaginary numbers. To overcome this, we will need to generate another plot in a co-ordinate system that incorporates imaginary numbers.\\~\

\end{frame}


%
% Frame 2j)
%

\begin{frame}
\frametitle{Dividing a real number by \(i\).}

What happens if we divide a real number by \(i\)?\\~\

Consider the following example, where the real number \(2\) is divided by \(i\);

\begin{align*}
\dfrac{2}{i} &= i^{4} \times \dfrac{2}{i}  \\
             &= \dfrac{i^{4} \times 2}{i}  \\
             &= i^{3}2
\end{align*}

Recall however, that multiplying a real number by \(i^{3}\) is the same as multiplying it by \(-i\).

\end{frame}


%
% Frame 2
%

\begin{frame}
\frametitle{3-d rotations using imaginary numbers?}

We have seen from the previous slides, that imaginary numbers allow us to easily perform 2-d rotations in the 2-d complex plane.\\~\

But what about if we wanted to perform 3-d rotations? Can imaginary numbers help with this problem at all? As it turns out, the answer is yes. 
The mathematician William Rowan Hamilton spent a good deal of his life on this problem, trying to come up with a solution -- and the solution which
he came up with are a concept in mathematics which are known as ``quaternions''.\\~\

but what are quaternions exactly? Quaternions -- along with octonions and sedenions, belong to a class of numbers known as ``hyper-complex numbers''.

\end{frame}


\begin{frame}
\frametitle{A gentle introduction to hyper-complex numbers.}

Take the complex plane which we discussed earlier. It is comprised of two number lines which are arranged relative to each other in an orthogonal fashion.
Now consider what would happen if we were to add another imaginary axis to it -- where this new axis would be
orthogonal to the original two axes which comprised the complex plane. We would now have a 3-d space which is comprised of one real axis and two imaginary axes.\\~\

Now let us take this a step further, and consider removing the real axis from this 3-d space. Furthermore, let us make up for the removal of the real
axis by adding another imaginary axis to the plane. We would now be back to having a 3-d space which is made up of three imaginary axes which are all 
orthogonal to each other.

\end{frame}



